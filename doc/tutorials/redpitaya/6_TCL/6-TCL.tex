\documentclass[12pt,oneside]{article}
\usepackage{makeidx,anysize,mflogo,xspace,float,epsfig,url}
\usepackage{amsmath,amsfonts,amssymb,a4wide} 
\usepackage[utf8]{inputenc}
%\usepackage[francais]{babel}
%\usepackage[french]{babel}
\urlstyle{sf}
%\usepackage{subcaption}
\usepackage{hyperref}
\usepackage{graphicx}
\usepackage{graphics}
\usepackage{float}
\usepackage{caption}
\usepackage{colordvi} %??
\usepackage{listings} 
\usepackage{subfigure}
\usepackage{subfloat}
\usepackage{xcolor}
\graphicspath{{./figures/}}
%\usepackage[labelsep=quad,indention=10pt]{subfig}
\definecolor{grey}{rgb}{0.95,0.95,0.95} % on définit la couleur grise
	% (c'est un gris très clair)
	\definecolor{red}{rgb}{1.0,0.0,0.0} 
	\definecolor{green}{rgb}{0.0,1.0,0.0}
	\definecolor{blue}{rgb}{0.0,0.0,1.0}
	\lstloadlanguages{bash,Java,C,C++,csh,make,sh}%%[Visual]Basic,xml}
	\lstset{frame=none,basicstyle=\footnotesize,breaklines,tabsize=2,captionpos=b,
		prebreak={\hbox{$\rightarrow$}},postbreak={\hbox{$\hookrightarrow$}},
		showstringspaces=false,backgroundcolor=\color{grey}\bfseries,
		keywordstyle=\color{blue},commentstyle=\color{green}\textit,
		stringstyle=\color{red}\ttfamily,abovecaptionskip=2pt,aboveskip=0pt,
		belowskip=0pt,belowcaptionskip=0pt,numbers=none,columns=fullflexible, backgroundcolor=\color{grey}}
%left,numberstyle=\footnotesize,
%		stepnumber=2,numbersep=1pt}

\begin{document}


\begin{center}
{\bf \Large Redpitaya TCL programming} \\ \ \\
G. Goavec-M\'erou, J.-M Friedt \\ \ \\ \today
\end{center}

This tutorial is a sequel to the second one on which it is based. In addition to 
copying the ADC inputs to the DAC outputs, we wish (Fig. \ref{obj}) to {\bf mix} the stream of 
data coming from the ADC with a local numerically controlled oscillator ({\bf NCO}) and finally
filter ({\bf FIR}) before sending back to the DAC and the processing system
(Fig. \ref{fin}). In this tutorial, all such tasks will be achieved by programming in TCL rather than
using the graphical user interface of Vivado.

\begin{figure}[h!tb]
\begin{center}
\includegraphics[width=\linewidth]{tutorial6}
\caption{Top: general objective of the frequency transposition of the input signal, followed by
filtering.}
\label{obj}
\end{center}

\includegraphics[width=\linewidth]{objective.pdf}
\caption{Schematic of the objective and final processing chain described in this document}
\label{fin}
\end{figure}

Starting from an existing TCL script, we first consider adding our processing blocks using the 
Vivado TCL commands as described at \url{https://www.xilinx.com/support/documentation/sw_manuals/xilinx2019_2/ug835-vivado-tcl-commands.pdf}.
The second part of the tutorial will wrap the Xilinx Vivado commands with custom OscimpDigital commands
to make the script compatible with the Altera/Intel synthesis tool.

\section{Initial TCL script}

Starting from {\tt tutorial3.tcl} in {\tt 3-PLPS/Design} which presents how to acquire data 
from the ADC and split the stream to send both the a data2ram and the DAC, we now wish to 
add the NCO + mixer + FIR filter.

The processor is instanciated:
{\footnotesize
\begin{verbatim}
startgroup
set ps7 [ create_bd_cell -type ip \
	-vlnv xilinx.com:ip:processing_system7:5.5 ps7 ]
set_property -dict [list CONFIG.PCW_IMPORT_BOARD_PRESET $board_preset ] $ps7
apply_bd_automation -rule xilinx.com:bd_rule:processing_system7 \
	-config {make_external "FIXED_IO, DDR" Master "Disable" Slave "Disable" } \
	$ps7
endgroup
\end{verbatim}
}

\noindent
followed by the ADC and DAC

{\footnotesize
\begin{verbatim}
set converters [ create_bd_cell -type ip -vlnv ggm:cogen:redpitaya_converters:1.0 converters]
set_property -dict [ list CONFIG.ADC_SIZE $ADC_SIZE \
	CONFIG.ADC_EN true \
	CONFIG.DAC_EN true] $converters
source $repo_path/redpitaya_converters/redpitaya_converters.tcl
connect_bd_intf_net [get_bd_intf_pins $converters/phys_interface] \
	[get_bd_intf_ports phys_interface_0]
\end{verbatim}
}

The streams are dupplicated to split between DAC and PS input

{\footnotesize
\begin{verbatim}
# dupplDataA
set dupplDataA [create_bd_cell -type ip -vlnv ggm:cogen:dupplReal_1_to_2:1.0 dupplDataA]
set_property -dict [ list CONFIG.DATA_SIZE $ADC_SIZE] $dupplDataA
connect_bd_intf_net [get_bd_intf_pins $converters/dataA_out] \
	[get_bd_intf_pins dupplDataA/data_in]

# dupplDataB
set dupplDataB [create_bd_cell -type ip -vlnv ggm:cogen:dupplReal_1_to_2:1.0 dupplDataB]
set_property -dict [ list CONFIG.DATA_SIZE $ADC_SIZE] $dupplDataB
connect_bd_intf_net [get_bd_intf_pins $converters/dataB_out] \
	[get_bd_intf_pins dupplDataB/data_in]
\end{verbatim}
}

Since in all Redpitayas the DAC is 14 bit but some Redpitayas are fitted with 16-bit ADC,
we shift to match bus sizes
{\footnotesize
\begin{verbatim}
# shifter is mandatory for redpitaya 16 : ADC is 16bits DAC 14bits
# shiftA_out
set shiftA_out [ create_bd_cell -type ip -vlnv ggm:cogen:shifterReal:1.0 shiftA_out]
set_property -dict [ list \
	CONFIG.SIGNED_FORMAT true \
	CONFIG.DATA_IN_SIZE $ADC_SIZE \
	CONFIG.DATA_OUT_SIZE 14] $shiftA_out
connect_bd_intf_net [get_bd_intf_pins dupplDataA/data2_out] \
	[get_bd_intf_pins $shiftA_out/data_in]
connect_bd_intf_net [get_bd_intf_pins shiftA_out/data_out] \
	[get_bd_intf_pins $converters/dataA_in]

# shiftB_out
set shiftB_out [ create_bd_cell -type ip -vlnv ggm:cogen:shifterReal:1.0 shiftB_out]
set_property -dict [ list \
	CONFIG.SIGNED_FORMAT true \
	CONFIG.DATA_IN_SIZE $ADC_SIZE \
	CONFIG.DATA_OUT_SIZE 14] $shiftB_out
connect_bd_intf_net [get_bd_intf_pins dupplDataB/data2_out] \
	[get_bd_intf_pins $shiftB_out/data_in]
connect_bd_intf_net [get_bd_intf_pins shiftB_out/data_out] \
	[get_bd_intf_pins $converters/dataB_in]
\end{verbatim}
}

Finally the data are sent to the PS
{\footnotesize
\begin{verbatim}
# convert realToComplex
set convertReal2Cplx [create_bd_cell -type ip -vlnv ggm:cogen:convertRealToComplex:1.0 convertReal2Cplx]
set_property -dict [ list CONFIG.DATA_SIZE $ADC_SIZE] $convertReal2Cplx
connect_bd_intf_net [get_bd_intf_pins dupplDataA/data1_out] \
	[get_bd_intf_pins convertReal2Cplx/dataI_in]
connect_bd_intf_net [get_bd_intf_pins dupplDataB/data1_out] \
	[get_bd_intf_pins convertReal2Cplx/dataQ_in]

# data2ram
set data1600 [create_bd_cell -type ip -vlnv ggm:cogen:dataComplex_to_ram:1.0 data1600]
set_property -dict [ list CONFIG.DATA_SIZE $ADC_SIZE \
						CONFIG.NB_INPUT {1} \
						CONFIG.NB_SAMPLE {4096}] $data1600
connect_bd_intf_net [get_bd_intf_pins convertReal2Cplx/data_out] \
	[get_bd_intf_pins data1600/data1_in]

apply_bd_automation -rule xilinx.com:bd_rule:axi4 \
    -config {Master "/ps7/M_AXI_GP0" Clk "Auto" } \
    [get_bd_intf_pins data1600/s00_axi]

connect_bd_net [get_bd_pins rst_ps7_125M/peripheral_reset] \
	[get_bd_pins $converters/adc_rst_i]
\end{verbatim}
}

The name of the interfaces are found using the {\tt oscimpDigital/fpga\_ip/tools/print\_businterfaces.py} and providing
as argument the directory name of the processing block in the {\tt fpga\_ip} directory. For example
{\tt ./tools/print\_businterfaces.py dupplComplex\_1\_to\_2}

{\footnotesize
\begin{verbatim}
./tools/print_businterfaces.py dupplComplex_1_to_2
component:
----------
name        : dupplComplex_1_to_2
VLNV        : ggm:cogen:dupplComplex_1_to_2:1.0
description : dupplComplex_1_to_2_v1_0

Parameters:
-----------
name: DATA_SIZE
  display name  : Data Size
    default value : 8
...
\end{verbatim}
}

\section{TCL script modifications}

The NCO AXI connection must be taken care of manually, following the same syntax as used in the data\_to\_ram block
with the apply\_bd\_automation command. Similarly, the NCO reset and clock must be connected to the converter clock
using 
{\footnotesize
\begin{verbatim}
connect_bd_net [get_bd_pins $converters/clk_o] \
	[get_bd_pins $datanco/ref_clk_i]
connect_bd_net [get_bd_pins $converters/rst_o] \
	[get_bd_pins $datanco/ref_rst_i]
\end{verbatim}
}
\noindent assuming the name of the NCO is {\tt datanco}.

See {\tt design/tutorial6.tcl} for the lines to be added to tutorial 2 to reach the expected result.

\section{OscimpDigital wrapper}

TCL is a verbose language and the commands are not consistent between Xilinx and Altera/Intel. Commands have
been implemented to OscimpDigital to add an abstraction layer and add the compatibility with any TCL-compatible
vendor (at the moment Xilinx and Altera/Intel).

\begin{itemize}
\item
{\tt add\_ip\_and\_conf}: adds an IP and configures parameters, using only the OscimpDigital name and no need to 
		provide the Xilinx VLNV -- Vendor, Library, Name, and Version -- identifier
\item {\tt connect\_proc}: connect to AXI bus. The argument is the least significant byte with respect to the 0x43C0000 base
address (start at 0x00000 and increment by 0x10000 for each new block to be consistent with Vivado's allocation of 64~KB blocks
for each IP),
\item {\tt connect\_proc\_clk}: connect the clock to the PS clock
\item {\tt connect\_proc\_rst}: connect the reset to the PS reset
\item {\tt connect\_intf}: connects interface to interface or signal to signal
\item {\tt connect\_to\_fpga\_pins}: same as Make external
\end{itemize}

Using these commands, the developer only focuses on connecting blocks together without
struggling with the details of each vendor implementation.

Similarly, the proposed Makefile automates all synthesis steps, whichever vendor is addressed
by a given platform: the project is synthesized using {\tt make project}.

Using these commands, the previous project becomes:

{\bf TODO}
\end{document}

